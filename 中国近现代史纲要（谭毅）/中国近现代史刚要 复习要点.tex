\documentclass{article}
\usepackage{CJK}
\usepackage{xypic}
\usepackage{amsfonts,amssymb}
\usepackage{multirow}
\usepackage{geometry}
\usepackage{graphicx}
\usepackage{listings}
\usepackage{listings}
\usepackage{lipsum}
\usepackage{courier}
\usepackage{indentfirst}


\linespread{1.2}
\geometry{left=3cm,right=2.5cm,top=2.5cm,bottom=2.5cm}
\lstset{basicstyle=\footnotesize\ttfamily,breaklines=true}
\lstset{numbers=left,frame=shadowbox,tabsize=4}
\setlength{\parindent}{2em}
\begin{document}
\begin{CJK}{UTF8}{song}
\title{中国近现代史刚要 复习要点}
\author {Africamonkey}
\date {考试日期:2017 年 1 月 4 日 9:30-11:30   考试地点:B203 }
\maketitle

\section{鸦片战争}
\subsection{鸦片战争失败的原因}
\subsubsection{社会制度的腐败}
1840年前后,中国封建社会逐步变成了半殖民地半封建社会。统治中国的清王朝,从皇帝到权贵,大多昏庸愚昧,不了解世界形势。许多官员贪污腐败,克扣军饷。不少将帅贪生怕死,临阵脱逃。他们大多害怕拥有坚船利炮的外国侵略者,甚至为了自己的私利,不惜出卖国家和民族的利益。 

鸦片战争中,禁言抗英有功的林则徐等人被革职查办,而主张对敌妥协的琦善等人反而得到重用。清政府害怕战争持续下去,会引发农民起义,因而急于向英国侵略者妥协,为此不惜割地赔款。
\subsubsection{闭关锁国,不明敌情}
清王朝狂妄自大,闭关自守,以天朝大国自诩,视外国为蛮夷,对国外的先进思想和科学技术,往往把它们当作“奇技淫巧”拒之门外。清政府不仅没有研究外国的机构,而且反对任何人这样做。虎门销烟后,英国政府决定调集军队,发动侵华战争,清政府竟毫无察觉,等到英军封锁珠江口,清政府仍对其侵略意图一无所知。
\subsubsection{经济技术的落后}
清军虽在人数上具有优势,但装备落后,火炮质量低劣,规格不一。多数清兵尚使用刀、矛、弓箭等冷兵器,火器也不过是鸟枪。而英军则普遍使用步枪,大炮则可打散弹,杀伤力强。
\subsubsection{战术落后}
清军将领军事思想保守落后,只会消极防御,不善于灵活击敌。清军将兵力分散到海岸线的各处,只会一线设防。英军多是利用清军设防的特点,用舰炮正面轰击。正确的方针应该是诱敌深入,然后发挥在本土作战的条件和兵力优势逐个消灭敌人。

\subsection{侵略给中国带来什么}
\subsubsection{政治方面}
清政府签订不平等条约,丧失部分主权。西方列强通过这些条约攫取在华特权,使中国的部分主权遭到破坏。

独立自主的封建中国开始沦为半殖民地半封建国家。鸦片战争前,中国是一个独立自主的国家。鸦片战争后,外国侵略者以不平等条约为掩护,逐步扩大对中国的侵略,把中国由独立自主的封建国家一步一步变为一个半殖民地半封建国家。

鸦片战争给中国的政治带来很大消极影响,但不容忽视的是,在一定意义上,鸦片战争对中国政治的发展有一定的促进作用。

清政府腐败、封闭、落后的大门被打开,封建制度逐渐解体。中国的大门被枪炮打开的同时也给中国带来了先进的世界文明,加速了封建制度的解体。
\subsubsection{经济方面}
白银外流,大量财富流往国外。鸦片战争中国战败,清政府被迫签订不平等条约,其中一项就是战争赔款,之后的赔款更是逐步增多。

掠夺资源,倾销商品和鸦片。丧权辱国的条约签订以后,西方列强在协定关税和领事裁判权的保护下,向中国倾销商品、掠夺资源、霸占市场,践踏了中国资本主义萌芽。

经济主权遭破坏,国家经济命脉受控于人。《南京条约》规定,英商进出口货物交纳的关税税率,中国政府必须同英国政府商定。中国的关税自主权被剥夺,经济主权遭到严重破坏。
\subsubsection{文化方面}
文化侵略。外国传教士来到中国,采用欺骗、强迫的手段霸占土地,建造教堂。有的还包揽词讼,包庇教徒中的不法分子,或者强迫中国教民抛弃中国传统礼俗。
\subsubsection{社会方面}
给人民带来沉重的负担。鸦片战争乃至以后每次战争都以失败告终。战败则必赔款,并且一次比一次多,这极大地加剧了人民的贫穷。

给人民带来巨大的灾难。从1840年鸦片战争以来,帝国主义列强发动了一次又一次的侵华战争。在历次侵华战争中,外国侵略者屠杀了大批中国人民。

\subsection{为什么说鸦片战争是近代史的起点}
第一,战争后中国的社会性质发生了根本性变化。随着外国资本主义的入侵,中国的封建社会逐步变成了半殖民地半封建社会。中国人民逐渐开始了反帝反封建的资产阶级民主革命。正因为如此,鸦片战争就称为中国近代史的起点。

第二,中国的发展方向发生变化,战前中国是一个没落的封建大国,封建制度已经腐朽,在缓慢地向资本主义社会发展;而鸦片战争后中国的民族资本主义不可能获得正常发展,中国也就不可能发展为成熟的资本主义社会,而最终选择了社会主义道路。

第三,社会主要矛盾发生变化,战前中国的主要矛盾是农民阶级与封建地主阶级的矛盾,而战后主要矛盾则包括农民阶级和地主阶级的矛盾及中华民族与外国殖民侵略者的矛盾,也就是社会主要矛盾复杂化。

第四,是革命任务发生变化,原先的革命任务是反对本国封建势力,战后则增加了反对外国殖民侵略的任务,革命的性质也由传统的农民战争转为旧民族主义革命。

\section{洋务运动}
\subsection{洋务运动的性质}
洋务运动既是地主阶级的一次自救运动,同时又是地主阶级所进行的一次改革运动。 

首先,在内忧外患的局势下,在镇压农民起义中崛起的部分地主官僚,借助洋枪洋炮,镇压了农民起义;同时又试图通过练兵,兴办军事工业和民用企业来“求强”、“求富”,以挽救清王朝。因此,洋务运动是一次地主阶级的自救活动。

其次,洋务派在不触动封建统治的基础上,实施了一些变革,引进西方科技,兴办近代企业,改革传统模式,这与当时世界资本主义发展的潮流是一致的,客观上有利于资本主义的发展,也是中国现代化过程中的一个重要环节。所以是一次改革运动。

\subsection{洋务运动失败的原因}
\subsubsection{洋务运动具有封建性}
洋务运动的指导思想是“中学为体,西学为用”,企图以吸收西方近代生产技术为手段,来达到维护和巩固中国封建统治的目的,这就决定了它必然失败的命运。因为新的生产力是同封建主义的生产关系及其上层建筑是不相容的。他们既要发展近代企业,却又采取垄断经营、侵吞商股等手段压制民族资本;既想培养洋务人才,又不愿改变封建科举制度。
\subsubsection{洋务运动对外国具有依赖性}
洋务运动进行之时,清政府已与西方国家签订了一批不平等条约,西方列强正是依据种种特权,从政治、经济等各方面加紧对中国的侵略和控制,它们并不希望中国真正富强起来。而洋务派官员却一再主张对外“和戎”,其所兴办的企业一切依赖外国,他们企图依赖外国来达到“自强”、“求富”的目的,无异于与虎谋皮。
\subsubsection{洋务运动的管理具有腐朽性}
洋务派所创办的一些新式企业虽然具有一定的资本主义性质,但其管理基本上仍是封建衙门式的。洋务派所办的军事工业完全由官方控制,经营不讲效益,造出的枪炮、轮船往往质量低下。即使是官商合办和官督商办的民用企业,其管理大多也是由政府“专派大员,用人理财悉听调度”,商人没有多少发言权,还要承担企业的亏损。企业内部极其腐败,充斥着营私舞弊、贪污受贿、挥霍浪费等官场恶习。

\subsection{洋务运动失败的教训}
一、在封建主义统治下的中国发展资本主义是不现实的。作为新的生产力的民族资本主义同封建主义统治下的生产关系以及上层建筑是相互排斥的,想要在中国发展资本主义,维护民族资本,培养新的适用性人才,就必须要从根本上改变中国的制度,革除封建主义,甚至是推翻清朝政府的统治。

二、作为侵略者的西方国家从本质上来说是不希望中国富强起来的,因为如果中国强大了,那么他们就无法在从中国获利。为了自身的利益着想,他们也不可能真正的帮助中国,让中国强大。所以在学习西方的先进事物时,一定要能够有所分辨,切不可照搬,也不能太过依赖西方国家。中国需要的是真正意义上的独立自主,新式的企业也需要新的管理体制,所以作为统治者,必须要给予商人一定的发言权,不要过多干预商业经济的发展。

\section{辛亥革命}

\subsection{辛亥革命失败的原因}
从根本上说,是因为在帝国主义时代,在半殖民地半封建的中国,资本主义的建国方案是行不通的。尽管当时先进的中国人真诚地希望把中国建设称为资产阶级共和国,但是,帝国主义决不允许中国建立一个独立、富强的资产阶级共和国,从而使自己失去中国这个占世界人口四分之一的剥削、奴役的对象。因此,它们用政治、外交、军事、经济、财政等各种手段来破坏、干涉中国革命,扶植并支持它们的代理人袁世凯来夺取政权。帝国主义与以袁世凯为代表的大地主大买办势力及旧官僚、立宪派一起勾结起来,从外部和内部绞杀了这场革命。

\subsubsection{没有提出彻底的反帝反封建的革命纲领}
他们没有明确提出反帝的口号,甚至幻想以妥协退让来换取帝国主义对中国革命的承认和支持。他们只反复强调反满和建立共和政体,并没有认识到必须反对整个封建统治阶级,致使一些汉族旧官僚、旧军官也混入革命的营垒。受当时政治局势的左右和妥协退让思想的支配,革命党人最后甚至还把政权拱手让给了袁世凯。
\subsubsection{不能充分发动和依靠人民群众}
由于中国民族资产阶级同封建势力有千丝万缕的联系,因而不敢依靠反封建的主力军农民群众。在革命的过程中,资产阶级革命派虽然也曾经联合新军和会党,从而在一定程度上动员了群众的力量,但在清政府被推翻之后,他们便把群众抛弃了。他们不但不去领导农民进行反封建的斗争,反而指责农民“行为越轨”,并派兵加以镇压。正因为中国民主革命的主力军农民没有被动员起来,这个革命的根基就显得相当单薄。正如毛泽东所说,国民革命需要一个大的农村变动。辛亥革命没有这个变动,所以失败了。
\subsubsection{不能建立坚强的革命政党,作为团结一切革命力量的强有力的核心}
同盟会内部的组织比较松懈,派系纷杂,缺乏一个统一和稳定的领导核心。
\subsection{为什么民主共和在中国难以实现}
一、民族资本主义基础薄弱,无力支撑共和国大厦。经济基础决定上层建筑。一种政治制度的产生和存在需要与之相适应的经济基础。民主共和制作为资产阶级政权,客观上要求资本主义有相当程度的发展。在中国,民族资本主义从诞生之日起,就受到帝国主义和封建主义的双重压迫,难以得到迅速发展。但从总体上看,它始终没有在中国社会经济生活中占据主导地位。经济力量的弱小和民族资产阶级成长的特殊环境又决定了资产阶级政治上十分软弱,容易妥协。

二、当时中国还缺乏建立民主政治的有利环境。辛亥革命虽然推翻了封建帝制,但没有完成反帝反封建的历史任务,特别是没有彻底打倒阻碍民主共和制建立的封建势力。

三、近代以来思想启蒙不足。长期以来,民主思想的传播主要在资产阶级及小资产阶级知识分子圈中进行的,并没有普及到人民群众中去。甚至一些革命派也未真正理解接受,而把推翻清王朝等同于民主共和的实现,以至于民主共和国诞生了,许多人还未意识到中国政治制度发生了根本性的变化,而仍然以为这不过是历史上的“改朝换代”。
\subsection{为什么资本主义的道路在中国走不通}
一、封建主义不愿意走资本主义道路。我国封建社会的历史长达两、三千年,在中国形成了世界上最完备也最顽固的封建主义生产关系。虽然中国封建社会内的商品经济的发展,已经孕育着资本主义的萌芽,但封建势力为了巩固封建统治地位,维护其政治、经济利益,不允许中国发展资本主义。在近代中国,封建势力还与帝国主义相勾结,压迫中国资本主义的发展。

二、帝国主义决不允许中国建立一个独立、富强的资产阶级共和国,从而使自己失去中国这个占世界人口四分之一的剥削、奴役的对象。因此,它们用政治、外交、军事、经济、财政等各种手段来破坏、干涉中国革命,扶植并支持它们的代理人袁世凯来夺取政权。帝国主义与以袁世凯为代表的大地主大买办势力及旧官僚、立宪派一起勾结起来,从外部和内部绞杀了这场革命。
\subsection{改良和革命的关系}
改良与革命,是社会政治变革的两种方式。革命与改良虽不是一回事,但在变革社会制度方面两者往往可以相辅相成。改良是渐进方式,革命是突变方式。革命前如果没有改良,恐怕革命难以发生;革命后没有改良,革命的成果也不易巩固。那么,革命与改良究竟有什么样的关系呢?改良与革命都是历史进程中的合理形式,都是适应不同需要而采取的不同手段,它们都有其各自的特殊功能,互相矛盾又相互补充,不可相互取代,也不能简单规定谁高谁低。

在整个世界近代化的过程中,中国的是十分落后的国家,并因而受到西方资本主义列强的侵略控制。当时,中国最高的利益就是采取各种各样的方法和措施,加速中国近代化的进程,才能为摆脱帝国主义侵略,建立独立自主的新中国奠定坚实的基础。从中 国近代化发展过程来看,先从局部开始实行改良、改革、变革,是加速这个进程的主要措施和方法之一。比如戊戌变法、清末新政中的改良主张,已突破具体局部改革的界限,而对整个封建制度中的弊端提出改革、改良、补救的方法。他们的出发点是爱国、强国,他们的终极目标是推动社会政治经济文化的文明程度,加速中国近代化的进程。 这同样是促进历史进步的,而不是与历史前进的方向背道而驰。

但是,历史发展表明,社会的进步光靠改革、改良还不行。旧制度对新事物的容忍是十分有限的。当改革改良的趋势达到一定的极限不能前进时,就需要用革命的暴力手段解决问题,推翻旧的专制统治,建立一个新的社会制度。孙中山领导的辛亥革命推翻了清王朝,结束了中国两千年多年的封建君 主专制统治,为资本主义在中国的发展开辟了道路,但是最终成果被北洋军阀窃取。因此,革命虽然是新社会的接生婆,为新生的社会生产力和生产关系开辟道路,但是革命要求严格的内外部条件。只有条件具备了才能产生真正的革命,而不是名义上的,只有条件具备了,革命才能成功。它包括新的社会生产力、生产关系的生长发展,新的觉悟的阶级力量及其革命意识的形成,革命领导核心的出现等。在十九世纪末二十年代初,由于中国近代资本主义没有得到充分发展,当然也没有产生阵容强大的资产阶级,没有成熟的无产阶级。因此在那时很难出现成熟的资产阶级革命和无产阶级革命。

由此可见,革命对于推动社会进步,促进中国近代化是一种主要手段,但不是唯一手段。当革命条件成熟时,压制革命、害怕革命、杞人忧天显然是不对的;但当革命的条件不成熟时,其它的改良、变法同样可以,而且必须。那种不顾客观和可能,把革命当作唯一的至高无上的手段,贬低一切改良、改革的观点,显然是错误的。

在社会历史的发展过程中,革命是社会变革的动力,在一定的条件下,改良也可以起到某种变革社会的作用。在某一国家的近代化变革中,究竟是采取革命的方式,还是采取改良的方式,完全取决于这个国家的历史状况、社会政治、经济状况、阶级状况等现实国情。也就是说,一切以时间、地点、条件为转移。一个国家内部如果必须以革命的方式才能解决问题,而革命的条件又已具备,在这种情况下鼓吹改良以抵制、反对革命,就应该受到贬斥。反之,如果不需要以革命的方式来解决,且又不具备革命的条件,却硬要采取革命的方式,也是不可取的。

因此,革命和改良,既有互相矛盾的一面,又有互相依存、补充的一面。

\section{中国共产党的成立}

\subsection{中国革命新面貌}
第一,第一次提出了反帝反封建的民主革命的纲领,为中国人民指出了明确的斗争目标。分清敌友,这是革命的首要问题。以往的斗争之所以成效甚少,一个重要的原因,就在于不能团结真正的朋友,以攻击真正的敌人。对于这个在长时间里没有得到解决的问题,中国共产党成立不久,就给予了一个基本的解决。

第二,开始采取群众路线的方法。是不是相信群众、依靠群众,这时关系革命成败的一个大问题。以往的斗争之所以成效甚少,一个重要的原因,就在于未能充分地发动群众。这种情况,在中国共产党成立不久,也有了一个根本的改变。

第三,实行国共合作,掀起大革命高潮。中国共产党认识到,中国无产阶级虽是一个最有觉悟性和最有组织性的阶级,但是如果单凭自己一个阶级的力量,是不能取得胜利的。而要胜利,他们就必须在各种不同的情形下团结一切可能团结的革命的阶级和阶层,组织革命的统一战线。
\subsection{中国共产党领导中国革命取得胜利的基本经验}

中国共产党在领导人民革命的过程中,积累了丰富的经验,毛泽东指出:“统一战线,武装斗争,党的建设,是中国共产党在中国革命中战胜敌人的三个法宝,三个主要的法宝。”

\subsubsection{建立广泛的统一战线}
由于中国人民受到帝国主义、封建主义、官僚资本主义的严重压迫,在中国建立革命统一战线的群众基础是十分广泛的。建立广泛的统一战线,是坚持和发展革命的政治基础。
\subsubsection{坚持革命的武装斗争}
由于中国没有资产阶级民主,反动统治阶级凭借武装力量对人民实行独裁恐怖统治,革命只能以长期的武装斗争作为主要形式。中国的武装斗争实质上是工人阶级领导的农民战争。中国共产党必须深入农村,发动和武装农民,在农村建立革命根据地,以农村包围城市,才能逐步地争取革命的胜利。
\subsubsection{加强共产党自身的建设}
中国共产党遵循毛泽东建党学说,在长期的斗争实践中,把自己锻炼成立一个有纪律的、有马克思列宁主义理论的、采取自我批评方法的、联系人民群众的党,成为掌握统一战线和武装斗争这两个武器以实行对敌冲锋陷阵的英雄战士,成为全国各族人民拥戴的领导核心。

\section{抗日战争}

\subsection{中国在世界反法西斯战争(第二次世界大战)中的作用}
中国人民抗日战争从一开始就具有拯救人类文明、保卫世界和平的重大意义,是世界反法西斯战争的重要组成部分,中国战场是世界反法西斯战争的东方主战场。

中国的抗战牵制和削弱了日本的力量,使之不敢贸然北进,从而使苏联得以集中兵力对付德国,避免东西两面作战;同时也推迟了日本发动太平洋战争的时间,并使之在发动和进行战争时由于兵力不足而不能全力南进,从而减轻了美、英军队受到的压力。

中国坚持持久抗战,抗击和牵制着日本陆军主力,并为同盟国军队实施战略反攻创造了有利条件。美国总统罗斯福说:“假如没有中国,假如中国被打垮了,你想一想有多少师的日本兵可以因此调到其他方面来作战?”
\subsection{正面战场的地位和作用}

\subsubsection{战略防御阶段的正面战场}
第一,国民党正面战场是抗日的主力。国民政府组织了一系列的大规模会战。如淞沪战役、晋北忻口战役、徐州和武汉的战役,都给日军以沉重的打击,是中国抗日战争乃至世界反法西斯战争的一个重要组成部分。

第二,由于正面战场的顽强抵抗,粉碎了日本帝国主义在3个月灭亡中国的战略计划和“速战速决”的方针。消耗了日本的军事、经济实力,使其陷入长期战争的泥坑而不能自拔。使日军兵力分散,战线延长,为战略相持阶段的到来,起了决定性的作用。

第三,支援了中国共产党领导的解放区敌后战场的开辟,为敌后游击战争创造了有利条件。

第四,国民党中爱国官兵的抗战英雄业绩,振奋了民族精神,大长了中华民族的志气,促进了全国的团结和进步,坚定了中国军民抗战必胜的信念。

第五、“唤起了国际舆论的同情和支持”,扩大了中国抗战在国际上的影响力。
\subsubsection{战略相持阶段的正面战场}

抗战进入相持阶段后,由于日本侵华方针的改变,国民党在正面战场上采取消极对敌,避战观战,消极抗日的态度,确定了“防共、限共、溶共、反共”的反动方针,建立了“防共委员会”。蒋介石则运用一切办法尽力限制八路军、新四军的发展。此后将大量军队用于对付中共军队相继掀起三次反共高潮,限制全国抗日民主力量。在政治上和军事上基本上都是起消极作用的。由于实行消极抗日积极反共的反动方针,正面战场形势严重恶化,以至于出现豫湘桂战役这样的大溃败。正面战场的地位逐步下降到次要位置。

尽管国民党抗战的态度较初期消极,但还是坚持了抗战。在此时期钳制了大量的日军,并使敌人受到一定的消耗,在客观上配合了敌后战场的坚持和发展,减轻了敌后战场的压力。从战略上讲,正面战场起到了战略配合的作用。

中国军民以重大牺牲换来了国际尊重,美英等国先后放弃了租界和其它在华特权,中国基本取得了国际关系中的平等地位,终止了晚清以来的屈辱状况。

从整个抗战过程看,尽管国民党在后期消极抗战,积极反共,但国民党政府的基本利益和民族利益总体来说是一致的,它始终坚持抗日的方针,没有妥协投降,从而保证了抗战的最后胜利。8年抗战,国民党正面战场共举行过22次重大战役,歼灭日军100余万,国民党军队伤亡321万。国民党军队的广大爱国官兵曾经在前线与日本侵略者奋勇作战,不怕流血牺牲,表现了强烈的爱国主义精神。一切与日本侵略者浴血奋战的爱国官兵,都为中国抗日战争的胜利贡献了力量,都是值得全民族尊敬与纪念的。

\subsection{敌后抗战的地位和作用}

在抗日战争的初期和中期,游击战争被提到了战略的地位,具有全局性的意义。

在战略防御阶段,从全局看,国民党正面战场的正规战是主要的,敌后的游击战是辅助的。但是,游击战在敌后的广泛开展和敌后抗日根据地的开辟,迫使敌人不得不把用于进攻的兵力抽调回去保守其占领区,从而对阻止日军的进攻、减轻正面战场压力、使战争转入相持阶段起了关键性的作用。

在战略相持阶段,敌后游击战争称为主要的抗日作战方式。日军逐步将主要兵力用于打击敌后战场的人民军队,以保持和巩固其占领地。1939年至1940年,华北地区的日军出动千人以上对敌后抗日根据地的大“扫荡”就有109次,使用的总兵力达50万人以上。削弱敌人、壮大自己,逐步改变敌强我弱的态势、为实行战略反攻准备条件,这个任务主要是由人民军队进行的游击战来完成的。

\section{近现代史的启示}

第一,“资本主义道路在中国走不通,只有共产党才能救中国”。这是这一时期历史的最主要的观点。无论是资产阶级维新派,还是资产阶级革命派,最后都逃脱不了失败的命运。失败,是这一时期永恒的主题。从“百日维新”开始,中国资产阶级就走上了自强探索的道路,但是辛酉政变的发生以及日后所谓的“清末新政”的丑剧,都说明了,资产阶级改良的道路在中国走不通。随后,一孙中山为首的资产阶级革命派,更是以救天下为己任。他们发动了辛亥革命,推翻了中国俩千多年的封建君主专制制度,建立了资产阶级共和国。但是,由于资产阶级本身的软弱性,最终,还是在中外反动势力的扼杀下失败了。以及后来的一系列运动,如:护国运动,护法运动,最终都是失败了。直到1921年7月1日,中共的诞生,才改变了这种局面。

第二,“落后就要挨打”。我对中国近现代史本身就有一种抵触的心里,因为这是一段无限屈辱的历史,从1840年鸦片战争开始,中国就陷入了一个怪圈,逢战必败,或者是不败而败。堂堂羊羊打过,竟只被一个小小的大不列颠打倒。堂堂四亿中国中国人,竟只被区区俩千英国兵打败,这是历史的悲哀,是中国的悲哀啊。随后,中国就开始了任人欺凌的时期,法国,俄罗斯,甚至连日本都来欺负中国。

第三,“民的力量是伟大的”。阻止帝国主义灭亡中国和推动中国社会向前发展的根本力量在于人民,在于人民不屈不挠、再接再厉的反侵略反压迫的英勇斗争。反抗英国鸦片侵略和反抗英法联军侵略的两次民族战争,反抗清朝统治的太平天国农民战争,就是这种人民的伟大斗争。洋务运动不属于人民的反抗运动。它的历史地位、历史作用决不能和人民的伟大斗争相提并论。不过,从另一方面看,在洋务运动和人民斗争之间,又存在着曲折的联系,洋务运动可以说是民族战争和农民战争的副产品。封建统治阶级中一部分有识之士,被民族战争的炮火所惊醒,探索战争胜败的因由,利用农民战争所造成的新局势,顺应历史发展的潮流,缓慢地逐步朝资本主义方向挪动,在暗地里或客观上为中国的独立和进步积累着物质力量,因此,不能以其不属于人民的反抗运动而完全抹煞。

第四,“要走改革开放的道路”。旧时代的清朝,一向以上国自居,盲目自大,实行闭关锁国政策。炮吓破了胆,才有一些有识之士掀起了洋务运动,但是,洋务运动也之士学习西方先进的科学技术,而不从根本上改变中国的政治体制。倡导所谓的“中体西用”原则,结果可想而知。而从这开始,中国的大门从未真正的向西方打开过,只是在被迫的情况下打开一点。这种状况在一定情况下导致中国的落后。到后来,洪仁玕的《资政新篇》中虽然提到了要学习西方,但由于农民阶级的局限,不可能真正的去学习西方的资本主义制度。所以,近代的中国,一直跟不上西方的步伐。

\end{CJK}
\end{document}

