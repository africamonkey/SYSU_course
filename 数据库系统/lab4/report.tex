% XeLaTeX

\documentclass{article}
\usepackage{ctex}
\usepackage{xypic}
\usepackage{amsfonts,amssymb}
\usepackage{multirow}
\usepackage{geometry}
\usepackage{graphicx}
\usepackage{listings}
\usepackage{lipsum}
\usepackage{courier}
\usepackage{fancyvrb}
\usepackage{etoolbox}


\linespread{1.2}
\geometry{left=3cm,right=2.5cm,top=2.5cm,bottom=2.5cm}

\makeatletter
\patchcmd{\FV@SetupFont}
  {\FV@BaseLineStretch}
  {\fontencoding{T1}\FV@BaseLineStretch}
  {}{}
\makeatother

\lstset{basicstyle=\small\fontencoding{T1}\ttfamily,breaklines=true}
\lstset{numbers=left,frame=shadowbox,tabsize=4}
%\lstset{extendedchars=false}
\begin{document}

\title{数据库系统实验4 \ 实验报告}
\author {数据科学与计算机学院 \ 计算机科学与技术 2016 级 \\ 王凯祺 \ 16337233}
\maketitle

\section{实验4 触发器实验}

\subsection{after 触发器}

在 takes 表上定义一个\emph{update 触发器},当成绩更新后,自动修改 students 表中的 tot\_cred ,以保持数据一致性。

\begin{lstlisting}[language=sql]
delimiter //
create trigger takes_cred_update after update on takes
for each row
begin
	if NEW.grade <> 'F' and NEW.grade is not null
		and (OLD.grade = 'F' or OLD.grade is null) then
		update student
		set tot_cred = tot_cred +
			(select credits
			from course
			where course.course_id = NEW.course_id)
		where student.ID = NEW.ID;
    end if;
    if OLD.grade <> 'F' and OLD.grade is not null
		and (NEW.grade = 'F' or NEW.grade is null) then
		update student
		set tot_cred = tot_cred -
			(select credits
			from course
			where course.course_id = OLD.course_id)
		where student.ID = OLD.ID;
    end if;
end;
\end{lstlisting}

在 takes 表上定义一个\emph{insert 触发器},当选课记录插入后,自动修改 students 表中的 tot\_cred ,以保持数据一致性。

\begin{lstlisting}[language=sql]
delimiter //
create trigger takes_cred_insert after insert on takes
for each row
begin
	if NEW.grade <> 'F' and NEW.grade is not null then
		update student
		set tot_cred = tot_cred +
			(select credits
			from course
			where course.course_id = NEW.course_id)
		where student.ID = NEW.ID;
    end if;
end;
\end{lstlisting}

在 takes 表上定义一个\emph{delete 触发器},当选课记录删除后,自动修改 students 表中的 tot\_cred ,以保持数据一致性。

\begin{lstlisting}[language=sql]
delimiter //
create trigger takes_cred_delete after update on takes
for each row
begin
    if OLD.grade <> 'F' and OLD.grade is not null then
		update student
		set tot_cred = tot_cred -
			(select credits
			from course
			where course.course_id = OLD.course_id)
		where student.ID = OLD.ID;
    end if;
end;
\end{lstlisting}

验证触发器 takes\_cred\_update 。

\begin{lstlisting}[language=sql]
select `grade`, `credits`
from takes natural join course 
where ID = '1000' and 
	course_id = '239' and 
	sec_id = '1' and 
	semester = 'Fall' and 
	year = 2006;
	
select * from student where ID = 1000;
\end{lstlisting}

\begin{lstlisting}
+-------+---------+
| grade | credits |
+-------+---------+
| C     |       4 |
+-------+---------+
1 row in set (0.03 sec)

+------+--------+------------+----------+
| ID   | name   | dept_name  | tot_cred |
+------+--------+------------+----------+
| 1000 | Manber | Civil Eng. |       39 |
+------+--------+------------+----------+
1 row in set (0.02 sec)
\end{lstlisting}

我们可以看到,学生1000已获学分39,某门课的成绩为C,学分为4。我们将这门课的成绩修改为F,观察触发器是否起作用。

\begin{lstlisting}[language=sql]
update `takes`
set grade='F'
where ID = '1000' and 
	course_id = '239' and 
	sec_id = '1' and 
	semester = 'Fall' and 
	year = 2006;
	
select * from student where ID = 1000;
\end{lstlisting}

运行结果:

\begin{lstlisting}
+------+--------+------------+----------+
| ID   | name   | dept_name  | tot_cred |
+------+--------+------------+----------+
| 1000 | Manber | Civil Eng. |       35 |
+------+--------+------------+----------+
1 row in set (0.04 sec)
\end{lstlisting}

在触发器的作用下,学号为 1000 的学分被修改。

\subsection{before 触发器}

在 takes 表上定义一个\emph{insert 触发器},当选课记录插入之前,先检查 prereq 表中该课程的前驱课程是否全部已修并合格。

\begin{lstlisting}[language=sql]
delimiter //
create trigger takes_prereq_insert before insert on takes
for each row
begin
	if exists(
		select 1
        from prereq
        where prereq.course_id = NEW.course_id and
			not exists (
				select 1
                from takes
                where prereq.prereq_id = takes.course_id and
					NEW.ID = takes.ID and
					takes.grade <> 'F' and
					takes.grade is not null
            )
    ) then
		signal sqlstate '45001' set message_text = "Prerequisite course not learned.";
    end if;
end;
\end{lstlisting}

验证触发器 insert 。

随便选一个 prereq 关系。

\begin{lstlisting}[language=sql]
SELECT * FROM lab.prereq limit 1;
\end{lstlisting}

\begin{lstlisting}
+-----------+-----------+
| course_id | prereq_id |
+-----------+-----------+
| 696       | 101       |
+-----------+-----------+
1 row in set (0.02 sec)
\end{lstlisting}

我们知道了 696 的前驱课程是 101 。

尝试添加一条记录:

\begin{lstlisting}[language=sql]
insert into takes values ('1000', '696', '1', 'Spring', 2002, null);
\end{lstlisting}

结果如下:

\begin{lstlisting}
ERROR 1644 (45001): Prerequisite course not learned.
\end{lstlisting}

由于 1000 未修读 101 课程,插入语句被拒绝。

尝试添加两条记录:

\begin{lstlisting}[language=sql]
insert into takes values ('1000', '101', '1', 'Fall', 2009, 'B');
insert into takes values ('1000', '696', '1', 'Spring', 2002, null);
\end{lstlisting}

结果如下:

\begin{lstlisting}
Query OK, 1 row affected (0.17 sec)
Query OK, 1 row affected (0.01 sec)
\end{lstlisting}

\subsection{删除触发器}

\begin{lstlisting}[language=sql]
drop trigger takes_prereq_insert;
\end{lstlisting}

\subsection{实验总结}

触发器的设计其实就是设计 select 语句,当满足一定条件时做特定的事情。

触发器本身的原理不难,但是语法却每种 SQL 语言都不一样。用 Mysql 需要查阅 Mysql 手册才能得知如何执行 rollback ,如何写条件语句等…… Mysql 没有 rollback 语句,但是可以 throw exception ,相当于中止执行。目前我还没试过 throw exception 之前执行过的语句是否会被 rollback ,目测是不行。

\end{document}
