% XeLaTeX

\documentclass{article}
\usepackage{ctex}
\usepackage{xypic}
\usepackage{amsfonts,amssymb}
\usepackage{multirow}
\usepackage{geometry}
\usepackage{graphicx}
\usepackage{listings}
\usepackage{lipsum}
\usepackage{courier}
\usepackage{fancyvrb}
\usepackage{etoolbox}


\linespread{1.2}
\geometry{left=3cm,right=2.5cm,top=2.5cm,bottom=2.5cm}

\makeatletter
\patchcmd{\FV@SetupFont}
  {\FV@BaseLineStretch}
  {\fontencoding{T1}\FV@BaseLineStretch}
  {}{}
\makeatother

\lstset{basicstyle=\small\fontencoding{T1}\ttfamily,breaklines=true}
\lstset{numbers=left,frame=shadowbox,tabsize=4}
%\lstset{extendedchars=false}
\begin{document}

\title{数据库系统实验6 \ 实验报告}
\author {数据科学与计算机学院 \ 计算机科学与技术 2016 级 \\ 王凯祺 \ 16337233}
\maketitle

\section{实验6 存储过程实验}

\subsection{无参数的存储过程}

定义一个存储过程,根据选课记录更新\emph{所有学生}的绩点。

\begin{lstlisting}[language=sql]
delimiter //
create procedure Proc_CalCred()
begin
	set SQL_SAFE_UPDATES = 0;
	update student
    set tot_cred = (
		select sum(credits)
        from takes natural join course
        where (takes.grade is not null) and takes.grade <> 'F' and student.ID = takes.ID
    );
    set SQL_SAFE_UPDATES = 1;
end //
delimiter ;
\end{lstlisting}

执行存储过程 Proc\_CalCred 。

\begin{lstlisting}[language=sql]
CALL `lab`.`Proc_CalCred`();
\end{lstlisting}

\subsection{有参数的存储过程}

定义一个存储过程,根据选课记录更新\emph{指定学生(学号)}的绩点。

\begin{lstlisting}[language=sql]
delimiter //
create procedure Proc_CalCred4Student(stu varchar(5))
begin
	update student
    set tot_cred = (
		select sum(credits)
        from takes natural join course
        where (takes.grade is not null) and takes.grade <> 'F' and student.ID = takes.ID
    )
    where student.ID = stu;
end //
delimiter ;
\end{lstlisting}

执行存储过程。

\begin{lstlisting}[language=sql]
CALL `lab`.`Proc_CalCred4Student`('1000');
\end{lstlisting}

\subsection{有局部变量的存储过程}

定义一个存储过程,根据选课记录更新\emph{指定学生(姓名)}的绩点。

\begin{lstlisting}[language=sql]
delimiter //
create procedure Proc_CalCred4Student_2(stu_name varchar(20))
begin
	declare idkey varchar(5);
    select ID into idkey
    from student
    where name = TRIM(stu_name);
	update student
    set tot_cred = (
		select sum(credits)
        from takes natural join course
        where (takes.grade is not null) and takes.grade <> 'F' and student.ID = takes.ID
    )
    where student.ID = idkey;
end //
delimiter ;
\end{lstlisitng}

执行存储过程。

\begin{lstlisting}[language=sql]
CALL `lab`.`Proc_CalCred4Student_2`('Manber');
\end{lstlisting}

查看存储过程执行结果。

\begin{lstlisting}[language=sql]
select tot_cred from student where name = 'Manber';
\end{lstlisting}

\subsection{有输出参数的存储过程}

定义一个存储过程,根据选课记录更新\emph{指定学生(姓名)}的绩点。

\begin{lstlisting}[language=sql]
delimiter //
create procedure Proc_CalCred4Student_3(stu_name varchar(20), out cred decimal(3,0))
begin
	declare idkey varchar(5);
    select ID into idkey
    from student
    where name = TRIM(stu_name);
	update student
    set tot_cred = (
		select sum(credits)
        from takes natural join course
        where (takes.grade is not null) and takes.grade <> 'F' and student.ID = takes.ID
    )
    where student.ID = idkey;
    select tot_cred into cred
    from student
    where ID = idkey;
end //
delimiter ;
\end{lstlisting}

执行存储过程。

\begin{lstlisting}[language=sql]
CALL `lab`.`Proc_CalCred4Student_3`('Manber', @a);
select @a;
\end{lstlisting}

结果如下:

\begin{lstlisting}
+------+
| @a   |
+------+
|   42 |
+------+
1 row in set (0.01 sec)
\end{lstlisting}

检查下列 SQL 语句执行结果与上述结果是否一致。

\begin{lstlisting}[language=sql]
select sum(credits)
from takes natural join course natural join (select ID, name from student) as stu
where (takes.grade is not null) and takes.grade <> 'F' and stu.ID = takes.ID and stu.name = 'Manber';
\end{lstlisting}

结果如下:

\begin{lstlisting}
+--------------+
| sum(credits) |
+--------------+
|           42 |
+--------------+
1 row in set (0.01 sec)
\end{lstlisting}

\subsection{删除存储过程}

\begin{lstlisting}[language=sql]
drop procedure `Proc_CalCred4Student_3`;
\end{lstlisting}

执行存储过程。

\begin{lstlisting}[language=sql]
CALL `lab`.`Proc_CalCred4Student_3`('Manber');
\end{lstlisting}

执行结果

\begin{lstlisting}
ERROR 1305 (42000): PROCEDURE lab.Proc_CalCred4Student_3 does not exist
\end{lstlisting}

\subsection{实验总结}

存储过程实际上是将重复性很高的一些操作,封装到一个存储过程中,从而简化了 SQL 的调用。对于大量的查询操作,可以减少 SQL 语句的传输,从而减少流量。存储过程的接口都是统一的,并且不会暴露数据库的结构,能确保数据的安全。

\end{document}
