\documentclass{article}
\usepackage{CJK}
\usepackage{xypic}
\usepackage{amsfonts,amssymb}
\usepackage{multirow}
\usepackage{geometry}
\usepackage{graphicx}
\usepackage{listings}
\usepackage{listings}
\usepackage{lipsum}
\usepackage{courier}


\linespread{1.2}
\geometry{left=3cm,right=2.5cm,top=2.5cm,bottom=2.5cm}
\lstset{basicstyle=\footnotesize\ttfamily,breaklines=true}
\lstset{numbers=left,frame=shadowbox,tabsize=4}
\begin{document}
\begin{CJK}{UTF8}{song}
\title{Project 1 技术报告}
\author {姓名:王凯祺 学号:16337233 班级:教务3班}
\maketitle

\section{技术要求}
实现一个图书馆管理系统(命令行程序),单一用户,支持以下操作:

1. 增加一本书

2. 删除一本书

3. 借阅一本书

4. 归还一本书

5. 查询一本书

6. 查询所有书籍

\section{实现思路}

本图书管理系统具有上述全部功能。其中,各功能的实现思路如下: \\
定义一个类 book ,表示一本书。 \\
增加一本书: 在 $std::vector<book>$ 中 push\_back 即可。 \\
删除一本书: 先遍历 vector 找到那本书,然后在 $std::vector<book>$ 中 erase 即可。 \\
借阅一本书: 先遍历 vector 找到那本书,然后标记被借阅。 \\
归还一本书: 先遍历 vector 找到那本书,然后取消标记被借阅。 \\
查询一本书: 遍历 vector 找到那本书,输出相关信息。 \\
查询所有书籍: 遍历 vector ,输出全部信息。

\section{对象设计}

\begin{lstlisting}[language=C++]
class book {
	public:
	book() {}
	book(int _borrowed);   //argument _borrowed represents whether the book is borrowed
	void borrow();         //mark the book as borrowed
	void return_();        //mark the book as returned
	std::string getname(); //get the name of the book
	bool getborrowed();    //get the borrowed mark of the book
	private:
	std::string name;
	bool borrowed;
};
\end{lstlisting}

\begin{lstlisting}[language=C++]
class lib {
	public:
	int print(std::string operation_type, int all = 0); //print the information of the book.
	//add a book and return success.
	void add(); //add a book
	void del(); //delete a book
	void borrow(); //borrow a book
	void return_(); //return a book
	void query(); //query a book.
	void query_all(); //query all books.
	private:
	std::vector<book> books;
};
\end{lstlisting}

\section{输入与输出}

按照提示输入输出。\\
在书名的输入中,允许出现的字符集为ASCII 32-127。

\end{CJK}
\end{document}
















